\subsection{Pseudocode van het parse algoritme}
\label{pseudo-code}
\begin{lstlisting}

	function Parse (string code)
	{
		/*Dit zal later nog anders ge�mplementeerd moeten worden
		  (dit is geen one-pass methode)*/
		code = splitsinregels (code);
		
		regel = 0;
		regel = springOverLegeRegels (code, regel);

		if (eersteKarakter == '>')
			-> dit is een folded scalar;

		else if (eersteKarakter == �|')
			-> dit is een blokje zonder folding;

		else if (eersteKarakters == '!!'
			-> expliciete aanduiding van het type: vb '!!seq'
			-> Parse dat type

		else if (eersteKarakter == '-')
		{
			-> dit is het begin van een sequence;
			sequence = new sequence;
			while (eersteKarakter == '-')
			{
				childNode = ingesprongen regels die volgen;
				sequence . addChlild (parse (childNode));

				-> Zoek het volgende '-' karakter
			}
		}

		else if (regel bevat een ":" of begint met "?")
		{
			-> Dit is een mapping

			mapping = new mapping;
			while (deze regel bevat ":" of begint met "?")
			{
				key = wat voor ":" staat of volgt na "?"
				childNode = ingesprongen regels die volgen na ":";
				mapping . addChild (parse (key), parse (childNode));

				-> Ga naar de volgende regel (bevat ook een ":")
			}
		}

		else
		{
			-> Dit is een scalar

			Bepaal het type scalar (datum, float, ...) en 
			geef het resultaat terug.
		}
	}

	function VerwijderIndentatie ()
	{
		-> Ga alle regels na, en vind de regel met de meeste voorgaande spaties
		-> Verwijder voor elke regel dit aantal spaties
		-> Geef het resultaat terug

		/* Nota: deze manier van werken laat geen one-pass 
			 processing toe, een andere oplossing is het aantal 
			 spaties voor de eerste regel te
			 bepalen en dat aantal overal van te verwijderen.
			 We moeten er dan  wel van uit gaan dat het 
			 bronbestand correcte YAML code bevat.*/
	}

\end{lstlisting}